\section{Introduction}

Computer information security is not an unfamiliar topic to most, but it is generally not something that the general public sees as a day-to-day issue. At the same time, technology has become an increasingly large portion of our everyday lives. With the advent of smart phones the ability to carry around a computer in one’s pocket has become not only a convenience, but a dependency for many people. Furthermore, the Internet has become a primary medium for social interaction and information gathering~\cite{KBK06}. In essence, there is little doubt that the Internet and computer technology is increasingly becoming a part of modern culture. Along with this comes an increase in the need for security.

In order to better understand the threats in such a world, it becomes increasingly more necessary to assume the shoes of the attacker. Security researchers are more easily able to understand the mind of the attacker if they are allowed to think like them by using their techniques and tools. Furthermore, using such tools gives researchers and administrators a means to break into their own systems so they can identify and fix vulnerabilities. 

It is with this mentality that we explore the use and implementation of an OS fingerprinter, a tool used by attackers to identify the operating system of a target machine over a network. We begin by establishing what an OS fingerprinter is and what role it plays in the attacker's process. Section 3 describes our method for implementing the program. This includes how we identify the operating systems, the implementation of the program itself, testing, and our findings. Section 4 describes other existing methods and programs for OS fingerprinting. Finally, we conclude by describing what we have learned.